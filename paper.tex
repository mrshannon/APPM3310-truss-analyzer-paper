\documentclass{article}

\title{APPM 3310 Project\\\Large{Static Analysis of Truss Systems}}
\author{Sammy Chang\and{}Ryan Marizza\and{}Michael Shannon}
\date{Friday, April 29, 2016}

\usepackage{hyperref}
\hypersetup{colorlinks=true, urlcolor=blue, linkcolor=blue, citecolor=blue}

\usepackage{amsmath}
\usepackage{amsfonts}
\usepackage{mathtools}
\usepackage{amssymb}
\usepackage{graphicx}
\usepackage{enumitem}

\usepackage{abstract}
\renewcommand{\abstractname}{}
\renewcommand{\absnamepos}{empty}
\renewcommand{\abstracttextfont}{\normalfont\small\bfseries}

\usepackage[citestyle=authoryear]{biblatex}
\addbibresource{paper.bib}

\begin{document}

\maketitle


\begin{abstract}
    Abstract goes here.
\end{abstract}


\section{Introduction}


\section{Mathematical Formulation}

Most of the following formulation follows directly from
\parencite{felippa2004}.

In this paper the accompanying tools use the direct stiffness method.  This
method is based on using known forces and displacements as well as the modulus
of elasticity of the structure members to back out displacements.  In this
way, it is unlike the direct force method which can only solve statically
determinate systems as the direct stiffness method can solve indeterminate
systems as well by properly splitting the stress among the truss members.

The direct stiffness method involves 8 steps.  The first two are purely
conceptual and thus are never really performed and the last one is a post
processing step performed after a solution is found.  These eight steps are
outlined below (taken from Figure 2.5 of \cite{felippa2004}).
\begin{enumerate}[noitemsep]
    \item Disconnection
    \item Localization
    \item Element Formation
    \item Globalization
    \item Merge
    \item Application of Boundary Conditions
    \item Solution
    \item Recovery of Derived Quantities
\end{enumerate}

The disconnection step is essentially just looking at each truss member
(element) individually while localization is just looking at member based
coordinate systems.  Therefore, this paper will skip directly to the first
step requiring actual work, \emph{element formation}.


\subsection{Element Formation}

The eventual goal is to find the master stiffness equation
\begin{equation}
    \mathbf{f}=K\mathbf{u} \label{eq:master_stiffness_equation}
\end{equation}
where $\mathbf{f}$ is the vector of external forces, $\mathbf{u}$ the vector
of displacements, and $K$ the linear relation between the two.  This equation
is in a global coordinate frame.  $K$ is $n\times{}n$ where $n$ is the number
of nodes (endpoint of one or more members) times the number of dimensions
which for the remainder of this document will be taken as 2.

Before the master stiffness equation can be written a relation between
displacement and external force for each element must be developed.  To do
this, the elements of the truss are treated as springs and therefore governed
by Hook's law.
\begin{equation}
    f=k_su \label{eq:hooks_law}
\end{equation}
where $f$ is the force, $u$ is the displacement, and $k_s$ is the stiffness of
the spring, which for an element of constant cross sectional area $A$, of
length $L$ and made of a material with elastic modulus $E$ is given by:
\begin{equation}
    k_s=\frac{EA}{L} \label{eq:stiffness_coefficient}
\end{equation}

By making the simplification that nodes cannot impart any torque on elements
the only equation governing an element is Hook's law.  In the element centered
coordinate frame the one dimensional Hook's law, Eq.~(\ref{eq:hooks_law}) is
enough.  However, in order to convert this into the global coordinate frame a
full dimensional equation is needed.  This equation will be designated as the
local stiffness equation:
\begin{equation}
    \overline{\mathbf{f}}=\overline{K}\overline{\mathbf{u}}
    \label{eq:local_stiffness_equation}
\end{equation}
This equation is in a member centric coordinate frame with the origin placed
on the first node of the element and the $x$-axis along the element, pointing
to the 2nd node.  The direction of all other axes are unimportant.  Therefore,
in the 2-dimensional case $\overline{K}$ becomes:
\begin{equation}
    \overline{K}=\frac{EA}{L}\begin{bmatrix*}[r]
        1&0&-1&0\\
        0&0&0&0\\
        -1&0&1&0\\
        0&0&0&0
    \end{bmatrix*}
\end{equation}
The 3-dimensional case can be easily obtained by adding 2 zero rows and 2 zero
columns.


\subsection{Globalization}

Before the global stiffness matrix can be formed
Eq.~(\ref{eq:local_stiffness_equation}) must be converted into the global
coordinate frame equation:
\begin{equation}
    \mathbf{f}^e=K^e\mathbf{u}^e
\end{equation}
which relates displacement to force for a single element $e$ in the global
coordinate frame.  Eq.~(\ref{eq:global_to_local}) shows the relation between the
global coordinate frame and the local coordinate frame.
\begin{equation}
    \overline{\mathbf{u}}^e=T^e\mathbf{u},\qquad
    \overline{\mathbf{f}}^e=T^e\mathbf{f}
    \label{eq:global_to_local}
\end{equation}



\subsection{Merge}


\subsection{Application of Boundary Conditions}


\subsection{Solution}


\subsection{Recovery of Derived Quantities}





\section{Examples and Numerical Results}


\section{Discussion}


\section{Conclusion}


\nocite{*}
\printbibliography{}


\end{document}
