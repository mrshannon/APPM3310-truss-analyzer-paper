\documentclass{article}

\title{APPM 3310 Project\\\Large{Static Analysis of Truss Systems}}
\author{Sammy Chang\and{}Ryan Marizza\and{}Michael Shannon}
\date{Friday, April 29, 2016}

\usepackage{hyperref}
\hypersetup{colorlinks=true, urlcolor=blue, linkcolor=blue, citecolor=blue}

\usepackage{amsmath}
\usepackage{amsfonts}
\usepackage{mathtools}
\usepackage{amssymb}
\usepackage{graphicx}
\usepackage{enumitem}
\usepackage{listings}

\usepackage{abstract}
\renewcommand{\abstractname}{}
\renewcommand{\absnamepos}{empty}
\renewcommand{\abstracttextfont}{\normalfont\small\bfseries}

\usepackage[citestyle=authoryear]{biblatex}
\addbibresource{paper.bib}

\begin{document}

\maketitle


\begin{abstract}
    Abstract goes here.
\end{abstract}


\section{Introduction}


\section{Mathematical Formulation}

Most of the following formulation follows directly from
\parencite{felippa2004ch2} and \parencite{felippa2004ch3}.

In this paper the accompanying tools use the direct stiffness method.  This
method is based on using known forces and displacements as well as the modulus
of elasticity of the structure members to back out displacements.  In this
way, it is unlike the direct force method which can only solve statically
determinate systems as the direct stiffness method can solve indeterminate
systems as well by properly splitting the stress among the truss members.

The direct stiffness method involves 8 steps.  The first two are purely
conceptual and thus are never really performed and the last one is a post
processing step performed after a solution is found.  These eight steps are
outlined below (taken from Figure 2.5 of \cite{felippa2004ch2}).
\begin{enumerate}[noitemsep]
    \item Disconnection
    \item Localization
    \item Element Formation
    \item Globalization
    \item Merge
    \item Application of Boundary Conditions
    \item Solution
    \item Recovery of Derived Quantities
\end{enumerate}

The disconnection step is essentially just looking at each truss member
(element) individually while localization is just looking at member based
coordinate systems.  Therefore, this paper will skip directly to the first
step requiring actual work, \emph{element formation}.


\subsection{Element Formation}

The eventual goal is to find the master stiffness equation
\begin{equation}
    \mathbf{f}=K\mathbf{u} \label{eq:master_stiffness_equation}
\end{equation}
where $\mathbf{f}$ is the vector of external forces, $\mathbf{u}$ the vector
of displacements, and $K$ the linear relation between the two.  This equation
is in a global coordinate frame.  $K$ is $n\times{}n$ where $n$ is the number
of nodes (endpoint of one or more members) times the number of dimensions
which for the remainder of this document will be taken as 2.

Before the master stiffness equation can be written a relation between
displacement and external force for each element must be developed.  To do
this, the elements of the truss are treated as springs and therefore governed
by Hook's law.
\begin{equation}
    f=k_su \label{eq:hooks_law}
\end{equation}
where $f$ is the force, $u$ is the displacement, and $k_s$ is the stiffness of
the spring, which for an element of constant cross sectional area $A$, of
length $L$ and made of a material with elastic modulus $E$ is given by:
\begin{equation}
    k_s=\frac{EA}{L} \label{eq:stiffness_coefficient}
\end{equation}

By making the simplification that nodes cannot impart any torque on elements
the only equation governing an element is Hook's law.  In the element centered
coordinate frame the one dimensional Hook's law, Eq.~(\ref{eq:hooks_law}) is
enough.  However, in order to convert this into the global coordinate frame a
full dimensional equation is needed.  This equation will be designated as the
local stiffness equation for element $e$:
\begin{equation}
    \overline{\mathbf{f}}^e=\overline{K}^e\overline{\mathbf{u}}^e
    \label{eq:local_stiffness_equation}
\end{equation}
This equation is in a member centric coordinate frame with the origin placed
on the first node of the element and the $x$-axis along the element, pointing
to the 2nd node.  The direction of all other axes are unimportant.  Therefore,
in the 2-dimensional case $\overline{K}^e$ becomes:
\begin{equation}
    \overline{K}^e=\frac{EA}{L}\begin{bmatrix*}[r]
        1&0&-1&0\\
        0&0&0&0\\
        -1&0&1&0\\
        0&0&0&0
    \end{bmatrix*} \label{eq:kbar}
\end{equation}
The 3-dimensional case can be easily obtained by adding 2 zero rows and 2 zero
columns.




\subsection{Globalization}

Before the master stiffness matrix can be formed,
Eq.~(\ref{eq:local_stiffness_equation}) must be converted into the global
coordinate frame equation:
\begin{equation}
    \mathbf{f}^e=K^e\mathbf{u}^e\label{eq:global_stiffness_equation}
\end{equation}
which relates displacement to force for a single element $e$ in the global
coordinate frame.  Let $\mathbf{x}$ be a vector in the global coordinate
system and $\overline{\mathbf{x}}$ be the equivalent vector in the element
local coordinate frame.  Then the relation between the two is given by
\parencite{olver2006}:
\begin{equation}
    S\overline{\mathbf{x}}=\mathbf{x}\qquad\text{where}\qquad
    S=\left\{\mathbf{v}_1,\mathbf{v}_2,\ldots,\mathbf{v}_n\right\}
\end{equation}
With the $\mathbf{v}$ vectors forming a basis for the element centered
coordinate system.  In particular, $v_1$ is the unit vector along the element.
If the other columns of $S$ are chosen such that $S$ is orthonormal then
$S^{-1}=S^{T}$.  Therefore, the reverse transformation is:
\begin{equation}
    \overline{\mathbf{x}}=S^{-1}\mathbf{x}=S^T\mathbf{x}
\end{equation}
Noting that $\mathbf{u}$ and $\mathbf{f}$ are stacked vectors that rotate
though each coordinate twice the appropriate transformation matrix for element
$e$ becomes:
\begin{equation}
    T^e=\begin{pmatrix}
        S^T&O\\O&S^T
    \end{pmatrix}
\end{equation}
Therefore, Eq.~(\ref{eq:global_to_local}) shows the relation between the
global coordinate frame and the local coordinate frame for the force and
displacement vectors.
\begin{equation}
    \overline{\mathbf{u}}^e=T^e\mathbf{u},\qquad
    \overline{\mathbf{f}}^e=T^e\mathbf{f}
    \label{eq:global_to_local}
\end{equation}
By combining Eq.~(\ref{eq:local_stiffness_equation},
\ref{eq:global_stiffness_equation}, \ref{eq:global_to_local}) the global
stiffness matrix (for element $e$), $K^e$ can be related to the local
stiffness matrix $\overline{K}^e$ by:
\begin{equation}
    K^e=(T^e)^T\overline{K}^eT^e
\end{equation}




\subsection{Merge}

The master stiffness equation Eq.~(\ref{eq:master_stiffness_equation}) relates
the displacement of every node, stored in $\mathbf{u}$ to every external force
in vector $\mathbf{f}$ using the linear transformation $K$.  The assembly of
each $K^e$ into $K$ is governed by \parencite{felippa2004ch3}:
\begin{enumerate}[noitemsep]
    \item Compatibility of displacements at any given node.
    \item Equilibrium of internal and external forces.
\end{enumerate}
The typical method of handling assembly of the global stiffness equations,
Eq.~(\ref{eq:global_stiffness_equation}), is to sum the forces and sum the
$K^e$ matrices by first expanding them to the size of $K$ by adding rows and
columns of zeros in the appropriate locations.  In the case of the computer
code that accompanies this document \parencite{shannon2016} the method in
section 3.5.1 of \cite{felippa2004ch3} is used.  This is know as
\emph{Assembly by Freedom Pointers}.  The algorithm is governed by
Eq.~(\ref{eq:global_to_master}) copied from \cite{felippa2004ch3} and fills $K$
by iterating through each element $e$ and then through the values in $K^e$
where $d=4$ in the 2-dimensional case and $d=6$ in the 3-dimensional case.
Let the number of elements in the truss be given by $m$.

\begin{minipage}{\textwidth}
\begin{equation*}
    K_{pq}=\sum_{e=1}^{m}K_{ij}^e
\end{equation*}
\begin{equation}
    \text{for}\label{eq:global_to_master}
\end{equation}
\begin{equation*}
    i=1,2,\ldots,d,\quad{}j=1,2,\ldots,d\qquad{}
    p=EFT^e(i),\quad{}q=EFT^e(j)
\end{equation*}
\vspace{0.1em}
\end{minipage}

The \emph{element freedom table} ($EFT^e$) is unique for each element $e$ and
gives the row/column indices where the values in $K^e$ are to be added into
$K$.  The specific algorithm to get $EFT^e$ is better explained in code and
can be found at \lstinline$libs/element_freedom_table.m$ in
\cite{shannon2016}.  In reality, $K$ is initialized to the zero matrix.  The
force vector $\mathbf{f}$ is simply constructed by summing the known (applied)
external forces in the proper indices which can also be obtained from the
\emph{element freedom table}.




\subsection{Application of Boundary Conditions}




\subsection{Solution}


\subsection{Recovery of Derived Quantities}





\section{Examples and Numerical Results}


\section{Discussion}


\section{Conclusion}


\nocite{*}
\printbibliography{}


\end{document}
